\setcounter{equation}{0}

\section*{Objective}

    Calculate the modulus of rigidity, $\eta$, of steel using a spring.


\section*{Concept}

    The time period of a spring-mass system in terms of the geometry of the spring and the modulus of rigidity of its material is given by:

    \beq \label{T}
        T = \frac{4 \pi}{r^2} \sqrt{\frac{N R^3 M}{\eta}}
    \eeq

    where $R$ is the radius of the spring, $N$ is the number of turns of the spring, $r$ is the radius of the wire that makes up the spring, and $M$ is the mass suspended from the spring.


\section*{Design}

    Use equation \eqref{T} to design an experiment that will allow you to calculate the modulus of rigidity $\eta$. Answer the following questions to quide your design.

    \begin{enumerate}[(i)]
        \item What is the independent variable in this equation? That is, what is the quantity you will vary?
        \item What is the dependent variable? That is, what will you measure every time you change the quantity above?
        \item What are the constants? The quantities you will keep fixed and measure only once.
        \item How will you measure each of these quantities? This determines your choice of apparatus.
        \item What will you plot on your graph?
        \item How will you determine the uncertainty in your measurements?
    \end{enumerate}

\section*{Requirements}

    You are required to

    \begin{enumerate}[(i)]
        \item Design the Experiment.
        \item Provide a List of Apparatus
        \item Write the Procedure
        \item Take measurements and record them in a Table
        \item Draw a graph
        \item Calculate the value of $\eta$ along with its uncertainty.
    \end{enumerate}
