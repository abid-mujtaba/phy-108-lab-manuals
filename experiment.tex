\section*{Objective}

    Use springs to investigate Hooke's Law, springs in series and parallel, and oscillations.

\section*{Apparatus}

    \begin{itemize}

        \item Stand
        \item Meter Rule
        \item Two Springs (with different spring constants)
        \item Slotted Mass Set
        \item Stopwatch

    \end{itemize}

\setcounter{section}{0}
\section{Single Spring}\label{single_spring}

    \subsection*{Objective}

        Calculate the spring constant of the two springs using Hooke's Law.

    \subsection*{Setup}

        Choose one of the springs and suspend it from the horizontal arm of the stand. Attach the mass hanger (no additional mass) to the bottom of the spring. This will cause the spring to stretch.% Once the spring comes to rest, use the meter rule to measure the new (extended) length of the spring. We will call this value $l_0$.
        %
        % \eline[0.5]
        % \begin{center}
        %     \begin{tabular}{|c|c|}
        %         \hline
        %             \tHW{50pt}{l_0}{\cm} & \makebox[50pt]{}\\
        %         \hline
        %     \end{tabular}
        % \end{center}
        % \eline[0.5]

    \subsection*{Procedure}

        Successively add additional mass to the hanger and each time measure the total length $l$ of the spring (once the spring has come to rest). Record you results in a table with the following format. \textit{Note}: $m$ is the total mass you have added to the spring (including that of the hanger).

        \eline[0.5]
        \begin{center}
            \begin{tabular}{| c | c |}
                \hline
                    \tH{m}{\si{\gram}} & \tH{l}{\cm}\\
                \hline
            \end{tabular}
        \end{center}

    \subsection*{Calculation}

        Draw a graph of $l$ (y-axis) vs. $m$ (x-axis) including steep and shallow fit lines. Using Hooke's Law $m g = k x$ calculate the the spring constant from the slope of the graph. Calculate the corresponding uncertainty in the value of the spring constant using the steep and shallow fit lines on your graph.

        Repeat the procedure and calculations for the second springs.

        Record your results in a table with the following format. Pay particular attention to the uncertainty and units of your results.

        \begin{center}
            \begin{tabular}{|c||c|}
                \hline
                \makebox[50pt]{$k_1$} & \makebox[50pt]{}\\
                \hline
                \makebox[50pt]{$k_2$} & \makebox[50pt]{}\\
                \hline
            \end{tabular}
        \end{center}


\section{Springs in Parallel}\label{parallel_springs}

    \subsection*{Objective}

        Verify that two springs in parallel have an effective spring constant given by
        \beq \label{eqn_parallel}
            k = k_1 + k_2
        \eeq

    \subsection*{Setup}

        Using the same two springs from section \ref{single_spring} attach them both to the horizontal arm of the stand, leaving some horizontal space between them. Now place a thin rod through the lower ends of both springs. Attach an empty mass hanger to the center of this thin rod. The mass of the hanger will cause the thin rod to stretch the two springs until they have equal stretched lengths.

    \subsection*{Procedure}

        Follow the procedure from section \ref{single_spring} to create a table and a graph of $l$ vs. $m$.

    \subsection*{Calculations}

        Use your graph to calculate the effective spring constant $k_{parallel}$ of the two parallel springs. Compare your answer to the expected value from equation \eqref{eqn_parallel} by calculating the percentage difference between the two values.

        The percentage difference between any two values $x_1$ and $x_2$ is defined as
        \beq \label{percentage_difference}
            \frac{| x_1 - x_2 |}{\frac{1}{2} (x_1 + x_2)} \times 100 \%
        \eeq


\section{Springs in Series}

    \subsection*{Objective}

        Verify that two springs in series have an effective springs constant given by
        \beq \label{series}
            \frac{1}{k} = \frac{1}{k_1} + \frac{1}{k_2}
        \eeq

    \subsection*{Setup}

        Using the same two springs attach one to the horizontal arm on the stand and attach the top of the other spring from the bottom of the first. Now attach the empty mass hanger to the bottom of the lower spring.

    \subsection*{Procedure}

        Follow the procedure from sections \ref{single_spring} and \ref{parallel_springs} to create a table and a graph of $l$ vs. $m$.

    \subsection*{Calculations}

        Use your graph to calculate the effective spring constant $k_{series}$ of the two springs in series. Compare your answer the expected value from equation \eqref{series} by calculating the percentage difference between the two values (using equation \eqref{percentage_difference}).
