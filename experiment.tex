\section{Objective}

    Calculate the value of $g$ (acceleration due to gravity) using a simple pendulum.

\section{Apparatus}

    \begin{itemize}

        \item String
        \item Mass (Bob)
        \item Stand
        \item Alligator Clip (\textit{optional})
        \item Stop-watch
        \item Meter-stick

    \end{itemize}

\section{Procedure}

    A simple pendulum consists of a mass (known as bob) which is attached to a string whose other end is held fixed (the pivot). The bob is then free to swing to and fro about the pivot. The time-period of these oscillations is given by
    %
    \beq \label{main}
        T = 2 \pi \sqrt{\frac{l}{g}}
    \eeq
    %
    where
    %
    \begin{itemize}
        \item $T$ is the Time Period
        \item $l$ is the length of the pendulum (from the pivot to the center of the bob)
        \item $g$ is the acceleration due to gravity
    \end{itemize}

    Our aim is to vary the length $l$ of the pendulum, by changing the length of the string, and measuring the corresponding value of the Time Period $T$. These observations will be used to calculate the value of $g$.

\subsection{Setup}

    Tie the bob to one end of the string. Make sure that the string is slightly longer than the height of the stand. Wrap the free end of the string around the top of the stand. Use the alligator clip to secure the top end of the string so that the length $l$ from the pivot to the center of the bob does not change as the pendulum oscillates.

\subsection{Tabulation}

    Record your data in a table with the following format.

    \begin{center}
        \begin{tabular}{| c || c | c | c | c | c |}

            \hline
            $l$ (\si{\centi\metre}) & $t_{10_1}$ (\si{\second}) & $t_{10_2}$ (\si{\second}) & $t_{10_3}$ (\si{\second}) & $t_{10_4}$ (\si{\second}) & $t_{10_5}$ (\si{\second}) \\

            \hline
               & & & & & \\
            \hline

        \end{tabular}
    \end{center}

    \eline
    For each value of $l$ we will measure the time for 10 oscillations, five times, giving us the five $t_{10_i}$. This table consists of the measurements we take in the experiment. From these measured values we calculate other derived quantities which will allow us to achieve our objective of calculating the value of $g$.

    \textit{It is recommended that you observe and right down all of the measurements first before you calculate the rest of the values. Using a pencil to write down the values will make it easy to fix inevitable mistakes.}

\subsection{Observation}

    Start with a length $l$ of $10$ \cm. Measure and record the five $t_{10_i}$, that is, measure the time for 10 oscillations five times. Then increase the value of $l$ by $10$ {\cm} and measure the five $t_{10_i}$ again.

    Keep increasing the value of $l$ by $10$ {\cm} until the length of the string exceeds the height of the stand and you are no longer able to swing the pendulum (since the bob starts to touch the table top).

\section{Calculations}

\subsection{Linearizing the Equation}

    The time period of the oscillations of a simple pendulum is given by equation \eqref{main}. This equation is not linear in the dependence of $T$ on $l$ since the variable $l$ appears inside a square root. To calculate the value of $g$ we must transform the equation in to a linear form which will allow us to plot a straight-line graph from the values we have measured.

    To that end we simply square both sides of \eqref{main} giving us
    %
    \beq
        T^2 = \frac{4 \pi^2}{g} l
    \eeq

    If we now consider the relationship between $T^2$ and $l$, it is linear. Compare this equation to that of the straight line $y = m x + c$ and you can immediately deduce that slope of the $T^2$ versus $l$ graph will be equal to
    %
    \beq \label{slope}
        \text{slope} = \frac{4 \pi^2}{g}
    \eeq
    %
    therefore we will be able to calculate the value of $g$ from the slope of the graph.

\subsection{Derived Quantities}

    Use your measurements (recorded in the table above) to calculate and record the derived quantities in a table with the following format.

    \eline
    \begin{center}
        \begin{tabular}{| c || c | c | c | c || c | c |}

            \hline
            \tHW{50pt}{l}{\cm} & \tHW{50pt}{\bar{t}_{10}}{\sec} & \tHW{50pt}{\Delta (t_{10})}{\sec} & \tHW{50pt}{T}{\sec} & \tHW{50pt}{\Delta (T)}{\sec} & \tHW{50pt}{T^2}{\si{\second \squared}} & \tHW{50pt}{\Delta (T^2)}{\si{\second \squared}} \\

            \hline
            & & & & & &\\
            \hline

        \end{tabular}
    \end{center}

    \eline
    where $\bar{t}_{10}$ is the mean/average of the five $t_{10_i}$ given by
    %
    % Note: The use of \foreach from the pgffor package to provide a loop for the four t_{10_i}
    \beq
        \bar{t}_{10} = \frac{1}{n} \sum\limits_{i = 1}^n t_{10_i} \, \bigg|_{n = 5} = \frac{1}{5} ( t_{10_1} \foreach \n in {2,...,5} { + t_{10_\n} } )
    \eeq

    $\Delta (t_{10})$ is the standard deviation (uncertainty) in the five values $t_{10_i}$ calculated using
    %
    \begin{equation}
        %
        % Note: The newcommand cannot be defined inside the 'aligned' environment otherwise it errors out
        \newcommand{\seed}[1]{(t_{10_#1} - \bar{t}_{10})^2}     % We define a command to create the ()^2 term for a specified value of i. Simplifies the equation below.
        %
        \begin{aligned}
            %
            \Delta (t_{10}) &= \sqrt{ \frac{1}{n (n - 1)} \sum\limits_{i=1}^n (t_{10_i} - \bar{t}_{10})^2} \, \bigg|_{n = 5} \\
                            &= \sqrt{ \frac{1}{5 \times 4} \big[ \seed{1} \foreach \n in {2,...,5}{ + \seed{\n}} \big] }
        \end{aligned}
    \end{equation}

    The time period $T$ (time for one oscillation) is calculated by dividing the mean time for $10$ oscillations $\bar{t}_{10}$ by $10$, that is
    %
    \beq
        T = \frac{\bar{t}_{10}}{10}
    \eeq

    The corresponding uncertainty $\Delta T$ is derived from the equation above and is given by
    %
    \beq
        \Delta T = \frac{\Delta (\bar{t}_{10})}{10}
    \eeq

    Finally, the uncertainty in the derived quantity $T^2$, denoted by $\Delta (T^2)$, comes from the uncertainty $\Delta T$ in $T$. It is calculated using
    %
    \beq
        \frac{\Delta (T^2)}{T^2} = 2 \frac{\Delta T}{T}
    \eeq
    %
    where the factor of $2$ comes from $T$ being raised to the power $2$ in the expression $T^2$.

\section{Graph}

    Now that we have values for $T^2$ versus $l$ (along with the uncertainty $\Delta (T^2)$ it is now time to draw a graph that will allow us to calculate the value of $g$.

    \begin{itemize}

        \item \textbf{Draw a linear graph of $T^2$ vs. $l$ including error bars.} \textit{Take special care with your choice of scale. Label all axes carefully.}

        \item \textbf{Use your graph to calculate the slope of the best fit line.} \textit{Don't forget to write down the units.}

        \item \textbf{Draw additional steep and shallow fit lines. Use these to calculate and note the uncertainty in the value of the slope.}

    \end{itemize}

\section{Results}

    \begin{enumerate}
        \item Use the calculated value of the slope, its uncertainty and equation \eqref{slope} to calculate the value of the acceleration due to gravity $g$ and its associated uncertainty.
        %
        \beq
            \frac{\Delta g}{g} = \frac{\Delta (\text{slope})}{\text{slope}}
        \eeq

        \item The actual value of $g$ is $981$ \si{\centi\meter \per \second \squared}. Compare your calculated value with this by calculating the percentage uncertainty. It is defined as
        %
        \beq
            \text{percentage uncertainty} = \frac{ |\text{actual value} - \text{measured value}| }{ \text{actual value} } \times 100 \%
        \eeq

        \item List the possible sources of error in this experiment.

    \end{enumerate}
