\section*{Objective}

   Calculate the viscosity $\eta$ of glycerin.

\section*{Apparatus}

   \begin{itemize}

      \item Graduated glass cylinder ($50 \cm$ long)
      \item Steel ball-bearings
      \item Conical Flask
      \item Stopwatch
      \item Micrometer
      \item Digital Balance
      \item Magnet attached to a string

   \end{itemize}

\section*{Procedure}

   The viscosity of a liquid can be measured using a viscometer. A viscometer consists of a graduated glass cylinder filled with the liquid in question, in our case glycerin. Small metallic spheres are dropped in to the liquid from the top. After falling sufficiently the spheres acquire terminal velocity. The viscosity of glycerin is given by 
   \beq \label{main}
      \eta = \frac{2}{9} \frac{(\rho - \sigma) g \, r^2 t}{d}
   \eeq
   %
   where
   %
   \begin{itemize}
      \item $\eta$ is the viscosity of glycerin
      \item $\rho$ is the density of the sphere
      \item $\sigma$ is the density of glycerin
      \item $g$ is the acceleration due to gravity
      \item $r$ is the radius of the sphere
      \item $d$ is the distance between two markers (AB = BC = $d$)
      \item $t$ is the time it takes for the sphere to cover the distance $d$
   \end{itemize}

\subsection*{Setup}

   The cylinder is filled with glycerin. The magnet tied to a string is gently lowered to the bottom of the cylinder such that a portion of the string hangs over the top edge. The purpose of the magnet is to aid in retrieving the metallic spheres we drop in to the glycerin. All of the spheres will fall to the bottom and become attached to the magnet. When one wants to retrieve the spheres simply pull on the string to raise the magnet and the attached spheres. \textit{Use tissue paper to handle the string so as to avoid getting glycerin on your hands or clothes. Don't worry, glycerin is perfectly harmless and washes out easily.}

   Place the conical flask on top of the cylinder. The metallic spheres are to be dropped through the conical flask to ensure that they fall through the center of the cylinder.

\subsection*{Choosing Markers}

   The first part of the experiment is determining where terminal velocity is achieved. Three points/markers (A, B, C) need to be chosen on the cylinder such that the distance AB = BC. For example if A is at the $15 \cm$ mark (on the scale attached to the cylinder) and B is at $30 \cm$ then C must be placed at $45 \cm$ so that AB = BC = $d$ = $15 \cm$.

   Initially choose A and B arbitrarily. Choose C to ensure that AB = BC. Now drop a sphere and measure the time $t_1$ and $t_2$ it takes for the sphere to move from A to B and B to C respectively. If terminal velocity has been achieved before the sphere reaches A we expect $t_1 = t_2$ (the two values should be within $0.5 \sec$). If terminal velocity is not achieved before A then we will observer $t_1 > t_2$ since the velocity at A will be larger than at B.

   If the two times don't match move the point A lower to allow the sphere to achieve terminal velocity. Move B and C accordingly. Repeat until the two times are approximately equal.

\subsection*{Constants}

   The density of the glycerin $\sigma$ is written on its container (bottle). Note it down in $\si[per-mode=symbol]{\gram \per \cubic \centi \metre}$. We also require the acceleration due to gravity. We will use the value $g = 981 \, \si[per-mode=symbol]{\centi\metre \per \squared \second}$

\subsection*{Dropping Spheres}





