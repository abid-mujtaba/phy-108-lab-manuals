\setcounter{equation}{0}

\section*{Objective}

    Calculate the modulus of rigidity, $\eta$, of steel using a spring.


\section*{Concept}

    The time period of a spring-mass system in terms of the geometry of the spring and the modulus of rigidity of its material is given by:

    \beq \label{T}
        T = \frac{4 \pi}{r^2} \sqrt{\frac{N R^3 M}{\eta}}
    \eeq

    where $R$ is the radius of the spring, $N$ is the number of turns of the spring, $r$ is the radius of the wire that makes up the spring, and $M$ is the mass suspended from the spring.


\section*{Requirements}

    Use equation \eqref{T} to design an experiment that will allow you to calculate the modulus of rigidity $\eta$.

    \begin{itemize}
        \item Use the
    \end{itemize}
