\section{Objective}

    Calculate the moment of inertia of a flywheel.

\section{Apparatus}

    \begin{itemize}

        \item Flywheel (mounted on a wall bracket)
        \item String
        \item Stopwatch
        \item Meter Rule
        \item Mass-Hanger set

    \end{itemize}

\section{Procedure}

    In this experiment we will be calculating the moment of inertia $I$ of a flywheel by using the Law of Conservation of Energy. We have shown that by equating the initial potential energy with the final Kinetic Energy and the Work Done against friction the moment of inertia can be calculated using the equation
    %
    \beq \label{I}
        I = \frac{N m}{N + n} \left( \frac{2 g h}{\omega^2} - r^2 \right)
    \eeq
    %
    where
    %
    \begin{itemize}

        \item $I$ is the moment of inertia
        \item $m$ is the mass of the weight attached to the flywheel using the string
        \item $h$ is the initial height of the weight
        \item $r$ is the radius of the axle of the flywheel (about which the string is wound)
        \item $n$ is the number of turns of the flywheel needed to wind the string to raise the weight to its initial position
        \item $N$ is the number of turns the flywheel makes after the weight has detached and it is brought to rest by friction
        \item $\omega$ is the angular speed of the flywheel right after the weight detaches
        \item $t$ is the time it takes

    \end{itemize}

    \eline
    We will be calculating $n$ and $\omega$ using
    %
    \beq \label{n}
        n = \frac{h}{2 \pi r}
    \eeq
    %
    and
    %
    \beq \label{omega}
        \omega = \frac{4 \pi N}{t}
    \eeq

    Our aim is to vary the mass $m$ of the weight (mass-hanger) and measure the corresponding values of $N$ and $t$. These will be used to calculate the corresponding values of the moment of inertia $I$ of the flywheel.

    \subsection{Setup}
