\section*{Objective}

   Calculate the viscosity $\eta$ of glycerin.

\section*{Apparatus}

   \begin{itemize}

      \item Graduated glass cylinder ($50 \, \cm$ long)
      \item Steel ball-bearings
      \item Conical Flask
      \item Stopwatch
      \item Micrometer
      \item Digital Balance
      \item Magnet attached to a string

   \end{itemize}

\section*{Procedure}

   The viscosity of a liquid can be measured using a viscometer. A viscometer consists of a graduated glass cylinder filled with the liquid in question, in our case glycerin. Small metallic spheres are dropped in to the liquid from the top. After falling sufficiently the spheres acquire terminal velocity. The viscosity of glycerin is given by 
   \beq \label{main}
      \eta = \frac{2}{9} \frac{(\rho - \sigma) g \, r^2 t}{d}
   \eeq
   %
   where
   %
   \begin{itemize}
      \item $\eta$ is the viscosity of glycerin
      \item $\rho$ is the density of the sphere
      \item $\sigma$ is the density of glycerin
      \item $g$ is the acceleration due to gravity
      \item $r$ is the radius of the sphere
      \item $d$ is the distance between two markers (AB = BC = $d$)
      \item $t$ is the time it takes for the sphere to cover the distance $d$
   \end{itemize}

\subsection*{Setup}

   The cylinder is filled with glycerin. The magnet tied to a string is gently lowered to the bottom of the cylinder such that a portion of the string hangs over the top edge. The purpose of the magnet is to aid in retrieving the metallic spheres we drop in to the glycerin. All of the spheres will fall to the bottom and become attached to the magnet. When one wants to retrieve the spheres simply pull on the string to raise the magnet and the attached spheres. \textit{Use tissue paper to handle the string so as to avoid getting glycerin on your hands or clothes. Don't worry, glycerin is perfectly harmless and washes out easily.}

   Place the conical flask on top of the cylinder. The metallic spheres are to be dropped through the conical flask to ensure that they fall through the center of the cylinder.

\subsection*{Choosing Markers}

   The first part of the experiment is determining where terminal velocity is achieved. Three points/markers (A, B, C) need to be chosen on the cylinder such that the distance AB = BC. For example if A is at the $15 \, \cm$ mark (on the scale attached to the cylinder) and B is at $30 \, \cm$ then C must be placed at $45 \, \cm$ so that AB = BC = $d$ = $15 \, \cm$.

   \tikzsetnextfilename{setup_experiment}

{
   % The definitions used by the two setup diagrams

% Define the height of the cylinder. We define it outside because this is also used by the flask
\def\Ch{5}
\def\Chw{0.5}

% Define the cylinder as an embeddable drawing
\newdrawing{\cylinder}{

      \def\hw{\Chw}      % Half width of cylinder
      \def\h{\Ch}         % Height of Cylinder
      \def\uw{0.2}      % Width of top-edge of cylinder

      \draw                         % Cylinder
         (-\hw,0) -- (-\hw,\h)
         (\hw,0) -- (\hw,\h)
         (-\hw,0) -- (\hw,0)

         (-\hw,\h) -- (-\hw-\uw,\h)
         (\hw,\h) -- (\hw + \uw,\h)
      ;
}

% Define the conical flask
\newdrawing{\flask}{

   \def\chw{0.1}       % Half-width of lower column
   \def\ch{0.75}           % Height of lower column

   \def\thw{0.8}        % Half-width of the top of the slanting bit
   \def\th{0.5}         % Height of the top of the slanting bit

   \coordinate (ol) at (-\chw, 0);
   \coordinate (or) at (\chw, 0);


   % We will place the origin horizontally center and vertically where the slanting part meets the lower column
   \draw
      (ol) -- (-\chw, -\ch)
      (or) -- (\chw, -\ch)

      (ol) -- (-\thw, \th)
      (or) -- (\thw, \th)
   ;
}


   \begin{center}
      \begin{tikzpicture}

         % The common drawing instructions for the two setup diagrams

\cylinder
\flask[yshift=\Ch-0.1]    % Place flask with its anchor (origin) just below the top of the cylinder

\def\ar{1}                % Draw the miniscus of the glycerin
% To draw an arc with center (0, \Ch), radius \ar and starting and ending angles -60 to -120 we use the 'shift' option.
% The arc command uses a starting point and then angles and radius and NOT the center
% To overcome this we define the starting point as the center shifted to the starting point using polar coordinates. The ability to shift using polar coordinates is very useful here.
% So [shift=(-60:\ar)] applied to the coordinates of the center gives the starting position.
% The second part containing the start and end angles and the radius of the arc are straight-forward
\draw ([shift=(-60:\ar)] 0,\Ch) arc (-60:-120:\ar);

\def\sr{0.05}
\draw (0,2.5) circle [radius=\sr];       % The falling sphere
\draw [->] (0,2.5-\sr) -- (0,2.2);


         \coordinate (lA) at (-\Chw, 3.25);
         \coordinate (lB) at (-\Chw, 1.75);
         \coordinate (lC) at (-\Chw, 0.25);

         \coordinate (cw) at (2 * \Chw, 0);

         % The three markers
         \draw [gray, text=black]           % The lines will be gray while the text will be black
            (lA) -- ++ (cw) node [right] {A}
            (lB) -- ++ (cw) node [right] {B}
            (lC) -- ++ (cw) node [right] {C}
         ;

         % Dimension lines for the distance 'd'
         \dimline[offset=-0.3, show extensions=false]{(lB)}{(lA)}{$d$}
         \dimline[offset=-0.3, show extensions=false]{(lC)}{(lB)}{$d$}

      \end{tikzpicture}
   \end{center}
}


   Initially choose A and B arbitrarily. Choose C to ensure that AB = BC. Now drop a sphere and measure the time $t_1$ and $t_2$ it takes for the sphere to move from A to B and B to C respectively. If terminal velocity has been achieved before the sphere reaches A we expect $t_1 = t_2$ (the two values should be within $0.5 \, \sec$). If terminal velocity is not achieved before A then we will observer $t_1 > t_2$ since the velocity at A will be larger than at B.

   If the two times don't match move the point A lower to allow the sphere to achieve terminal velocity. Move B and C accordingly. Repeat until the two times are approximately equal.

\subsection*{Tabulation}

   Record the position of the markers in a table with the following format. \textit{Don't forget to write down the units}.

   {
      \def\hs{\hspace{20pt}}

      \begin{ctable}{| c | c | c |}

         \hline
            \tH{A}{\hs} & \tH{B}{\hs} & \tH{C}{\hs}\\
         \hline
            & & \\
         \hline

      \end{ctable}
   }

   Record the constants in a table with the following format. \textit{Pay close attention to the units of $d$}.

   {
      \def\th{\tHW{50pt}}

      \begin{ctable}{| c | c | c |}

         \hline
            \th{d}{\cm} & \th{\sigma}{\gpcc} & \th{g}{\cmpss}\\
         \hline
            & & $981$\\
         \hline

      \end{ctable}
   }


   Record your observations in a table with the following format.

   {
      \def\th{\tHW{50pt}}

      \begin{ctable}{| c | c | c | c |}

         \hline
            \th{D}{\cm} & \th{m}{\si{\gram}} & \th{t_1}{\sec} & \th{t_2}{\sec}\\
         \hline
            & & &\\
         \hline

      \end{ctable}
   }
   where $D$ is the diameter of the sphere, $t_1$ is the time it takes for the sphere to move from A to B, and $t_2$ is the time it takes for it to move from B to C.

\subsection*{Constants}

   The density of the glycerin $\sigma$ is written on its container (bottle). Note it down in $\gpcc$. We also require the acceleration due to gravity. We will use the value $g = 981 \, \cmpss$

\subsection*{Observation}

   Choose 5 spheres with different diameters. For each sphere measure its mass $m$ using the digital balance, and its diameter $D$ using the micrometer. Note these down in your table.

   Now drop the sphere in to the glycerin using the conical flask placed on top of the cylinder. Use two stop-watches to measure the time $t_1$ it takes to move from A to B and time $t_2$ it takes to move from B to C. Note these down in the table as well.

\section*{Calculations}

   Use your measurements to calculate and record the derived quantities in a table with the following format

   {
      \def\th{\tHW{40pt}}       % Define the table header with fixed width for use in the table below

      \begin{ctable}{| c || c | c | c || c |}

         \hline
            \th{D}{\cm} & \th{r}{\cm} & \th{\rho}{\gpcc} & \th{t}{\sec} & \th{\eta}{\hspace{20pt}} \\
         \hline
            & & & & \\
         \hline

      \end{ctable}
   }
   %
   where
   {
      \def\vs{0.5\baselineskip}

      \beqcn
         r = \frac{D}{2}\\[\vs]
         \rho = \frac{m}{V} = \frac{3 \, m}{4 \pi r^3}\\[\vs]
         t = \frac{t_1 + t_2}{2}
      \eeqcn
   }

   Use \eqref{main} to calculate both the values and \textbf{units} of $\eta$.

\section*{Results}

   Having performed the experiment using 5 spheres you should now have 5 values for the viscosity. Calculate the average and standard deviation of these values to come up with a final value for the viscosity of glycerin.   
