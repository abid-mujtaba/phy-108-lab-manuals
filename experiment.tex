\section*{Objective}

    Use springs to investigate Hooke's Law, springs in series and parallel, and oscillations.

\section*{Apparatus}

    \begin{itemize}

        \item Stand
        \item Meter Rule
        \item Two Springs (with different spring constants)
        \item Slotted Mass Set
        \item Stopwatch

    \end{itemize}

\setcounter{section}{0}
\section{Single Spring}\label{single_spring}

    \subsection*{Objective}

        Calculate the spring constant of the two springs using Hooke's Law.

    \subsection*{Setup}

        Choose one of the springs and suspend it from the horizontal arm of the stand. Attach the mass hanger (no additional mass) to the bottom of the spring. This will cause the spring to stretch.

        \tikzsetnextfilename{single_spring}

\begin{center}
    \begin{tikzpicture}

        
% Define styles that will automatically draw the spring and slanting line platform objects
\tikzstyle{spring1}=[thick, decorate, decoration={aspect=0.5, segment length=2mm, amplitude=2mm, coil}]
\tikzstyle{spring2}=[thick, decorate, decoration={aspect=0.5, segment length=3mm, amplitude=2mm, coil}]

\tikzstyle{platform}=[fill, pattern=north east lines, draw=none]

\def\hww{0.25}      % Half Width of weight
\def\lv{1}          % Length of vector in FBD

% Define a 'newdrawing' that corresponds to the weight box
\newdrawing{\weight}{
    \draw (-\hww, 0) rectangle (\hww, {-2 * \hww}) node [midway] {$m$};        % Note: the use of [midway] to place the label midway between the starting and ending points of the rectangle
}

\def\ph{0.25}       % height of the slanted line rectangle that constitutes a platform


        \def\l{2}           % Length of spring
        \def\hw{0.75}

        \draw [platform] (-\hw,\l) rectangle (\hw,{\l + \ph});
        \draw [very thick] (-\hw,\l) -- (\hw,\l);       % Thick line underneath platform to which the spring is attached
        \draw [spring1] (0,\l) -- (0,0);

        \weight[]           % Draw the weight object using the command generated by the newdrawing macro

    \end{tikzpicture}
\end{center}


    \subsection*{Procedure}

        Successively add additional mass to the hanger and each time measure the total length $l$ of the spring (once the spring has come to rest). Record you results in a table with the following format. \textit{Note}: $m$ is the total mass you have added to the spring (including that of the hanger).

        \eline[0.5]
        \begin{center}
            \begin{tabular}{| c | c |}
                \hline
                    \tH{m}{\si{\gram}} & \tH{l}{\cm}\\
                \hline
            \end{tabular}
        \end{center}

    \subsection*{Calculation}

        Draw a graph of $l$ (y-axis) vs. $m$ (x-axis) including steep and shallow fit lines. Using Hooke's Law $m g = k x$ calculate the the spring constant from the slope of the graph. Calculate the corresponding uncertainty in the value of the spring constant using the steep and shallow fit lines on your graph.

        Repeat the procedure and calculations for the second springs.

        Record your results in a table with the following format. Pay particular attention to the uncertainty and units of your results.

        \begin{center}
            \begin{tabular}{|c||c|}
                \hline
                \makebox[50pt]{$k_1$} & \makebox[50pt]{}\\
                \hline
                \makebox[50pt]{$k_2$} & \makebox[50pt]{}\\
                \hline
            \end{tabular}
        \end{center}


\section{Springs in Parallel}\label{parallel_springs}

    \subsection*{Objective}

        Verify that two springs in parallel have an effective spring constant given by
        \beq \label{eqn_parallel}
            k = k_1 + k_2
        \eeq

    \subsection*{Setup}

        Using the same two springs from section \ref{single_spring} attach them both to the horizontal arm of the stand, leaving some horizontal space between them. Now place a thin rod through the lower ends of both springs. Attach an empty mass hanger to the center of this thin rod. The mass of the hanger will cause the thin rod to stretch the two springs until they have equal stretched lengths.

        \tikzsetnextfilename{parallel_springs}

\begin{center}
    \begin{tikzpicture}

        
% Define styles that will automatically draw the spring and slanting line platform objects
\tikzstyle{spring1}=[thick, decorate, decoration={aspect=0.5, segment length=2mm, amplitude=2mm, coil}]
\tikzstyle{spring2}=[thick, decorate, decoration={aspect=0.5, segment length=3mm, amplitude=2mm, coil}]

\tikzstyle{platform}=[fill, pattern=north east lines, draw=none]

\def\hww{0.25}      % Half Width of weight
\def\lv{1}          % Length of vector in FBD

% Define a 'newdrawing' that corresponds to the weight box
\newdrawing{\weight}{
    \draw (-\hww, 0) rectangle (\hww, {-2 * \hww}) node [midway] {$m$};        % Note: the use of [midway] to place the label midway between the starting and ending points of the rectangle
}

\def\ph{0.25}       % height of the slanted line rectangle that constitutes a platform


        \def\l{2}           % Length of spring
        \def\hw{1.5}
        \def\hws{1}

        \draw [platform] (-\hw,\l) rectangle (\hw,{\l+\ph});    % Using the [platform] modifier sets the fill pattern for the rectangle
        \draw [very thick] (-\hw,\l) -- (\hw,\l);       % Thick line underneath platform to which the spring is attached

        \draw [spring1] (-\hws,\l) -- (-\hws,0);
        \draw [spring2] (\hws,\l) -- (\hws,0);

        \draw [very thick] (-\hw, 0) -- (\hw, 0);
        \draw (0,0) -- (0,-0.25);

        \weight[yshift=-0.25]           % Draw the weight object using the command generated by the newdrawing macro

    \end{tikzpicture}
\end{center}


    \subsection*{Procedure}

        Follow the procedure from section \ref{single_spring} to create a table and a graph of $l$ vs. $m$.

    \subsection*{Calculations}

        Use your graph to calculate the effective spring constant $k_{parallel}$ of the two parallel springs. Compare your answer to the expected value from equation \eqref{eqn_parallel} by calculating the percentage difference between the two values.

        The percentage difference between any two values $x_1$ and $x_2$ is defined as
        \beq \label{percentage_difference}
            \frac{| x_1 - x_2 |}{\frac{1}{2} (x_1 + x_2)} \times 100 \%
        \eeq


\section{Springs in Series}

    \subsection*{Objective}

        Verify that two springs in series have an effective springs constant given by
        \beq \label{series}
            \frac{1}{k} = \frac{1}{k_1} + \frac{1}{k_2}
        \eeq

    \subsection*{Setup}

        Using the same two springs attach one to the horizontal arm on the stand and attach the top of the other spring from the bottom of the first. Now attach the empty mass hanger to the bottom of the lower spring.

        \tikzsetnextfilename{series_springs}

\begin{center}
    \begin{tikzpicture}

        
% Define styles that will automatically draw the spring and slanting line platform objects
\tikzstyle{spring1}=[thick, decorate, decoration={aspect=0.5, segment length=2mm, amplitude=2mm, coil}]
\tikzstyle{spring2}=[thick, decorate, decoration={aspect=0.5, segment length=3mm, amplitude=2mm, coil}]

\tikzstyle{platform}=[fill, pattern=north east lines, draw=none]

\def\hww{0.25}      % Half Width of weight
\def\lv{1}          % Length of vector in FBD

% Define a 'newdrawing' that corresponds to the weight box
\newdrawing{\weight}{
    \draw (-\hww, 0) rectangle (\hww, {-2 * \hww}) node [midway] {$m$};        % Note: the use of [midway] to place the label midway between the starting and ending points of the rectangle
}

\def\ph{0.25}       % height of the slanted line rectangle that constitutes a platform


        \def\la{2}           % Length of spring
        \def\lb{4}
        \def\hw{0.75}

        \draw [platform] (-\hw,\lb) rectangle (\hw,{\lb + \ph});
        \draw [very thick] (-\hw,\lb) -- (\hw,\lb);       % Thick line underneath platform to which the spring is attached

        \draw [spring2] (0,{\la + 0.2}) -- (0,\lb) node [midway, right=7] {$k_2$};

        % Draw a line with a small circle in the middle to show the connection between the two springs
        \draw (0,\la) -- (0,{\la + 0.2});
        \draw [fill=black] (0,{\la + 0.1}) circle [radius=0.05];

        \draw [spring1] (0,\la) -- (0,0) node [midway, right=7] {$k_1$};

        \weight[]

    \end{tikzpicture}
\end{center}


    \subsection*{Procedure}

        Follow the procedure from sections \ref{single_spring} and \ref{parallel_springs} to create a table and a graph of $l$ vs. $m$.

    \subsection*{Calculations}

        Use your graph to calculate the effective spring constant $k_{series}$ of the two springs in series. Compare your answer the expected value from equation \eqref{series} by calculating the percentage difference between the two values (using equation \eqref{percentage_difference}).


\section{Oscillating Spring}

    \subsection*{Objective}

        Calculate the spring constant of a spring using the Time Period of its oscillation as given by
        \beq \label{time_period}
            T = 2 \pi \sqrt{\frac{m}{k}}
        \eeq

    \subsection*{Setup}

        Choose a single spring. Attach it to the stand and suspend the mass hanger from its lower end.

        \tikzsetnextfilename{single_spring}

\begin{center}
    \begin{tikzpicture}

        
% Define styles that will automatically draw the spring and slanting line platform objects
\tikzstyle{spring1}=[thick, decorate, decoration={aspect=0.5, segment length=2mm, amplitude=2mm, coil}]
\tikzstyle{spring2}=[thick, decorate, decoration={aspect=0.5, segment length=3mm, amplitude=2mm, coil}]

\tikzstyle{platform}=[fill, pattern=north east lines, draw=none]

\def\hww{0.25}      % Half Width of weight
\def\lv{1}          % Length of vector in FBD

% Define a 'newdrawing' that corresponds to the weight box
\newdrawing{\weight}{
    \draw (-\hww, 0) rectangle (\hww, {-2 * \hww}) node [midway] {$m$};        % Note: the use of [midway] to place the label midway between the starting and ending points of the rectangle
}

\def\ph{0.25}       % height of the slanted line rectangle that constitutes a platform


        \def\l{2}           % Length of spring
        \def\hw{0.75}

        \draw [platform] (-\hw,\l) rectangle (\hw,{\l + \ph});
        \draw [very thick] (-\hw,\l) -- (\hw,\l);       % Thick line underneath platform to which the spring is attached
        \draw [spring1] (0,\l) -- (0,0);

        \weight[]           % Draw the weight object using the command generated by the newdrawing macro

    \end{tikzpicture}
\end{center}


    \subsection*{Procedure}

        Successively add additional mass to the hanger. For each new mass added, pull down the mass slightly and release. This will cause the mass to start performing vertical oscillations.

        Wait for the mass to oscillate at least twice. The next time the mass passes its equilibrium (middle) position going down start your stopwatch and count zero. The mass will move down, back up, will cross the equilibrium position going up, turn back around and then pass the equilibrium position going down.

        Only when it passes the equilibrium position going down does one complete oscillation occur. Only then should you advance your count to one. Keep counting until the mass has completed 20 oscillations at which point stop your stopwatch.

        Stop the mass. Repeat the process described above to get two more readings for $T_{20}$, the time for 20 oscillations.

        Record the total mass (including hanger) and $T_{20}$ (the time for 20 oscillations) in a table with the following format. \textit{Only fill in the first four columns which comprise the observations. You can fill out the rest of the columns after you have completed your observations and are starting your calculations.}

        \eline
        \begin{center}
           \begin{tabular}{| c | c | c | c || c | c || c | c |}
              \hline
              \tH{m}{\si{\gram}} & \tH{T_{20 \_ 1}}{\sec} & \tH{T_{20 \_ 2}}{\sec} & \tH{T_{20 \_ 3}}{\sec} & \tHW{40pt}{T}{\sec} & \tHW{40pt}{\Delta T}{\sec} & \tHW{40pt}{T^2}{\si{\second \squared}} & \tH{\Delta (T^2)}{\si{\second \squared}}\\
              \hline
              &&&&&&&\\
              \hline
           \end{tabular}
         \end{center}


   \subsection*{Calculations}

        Equation \eqref{time_period} is not linear between $T$ and $m$. To linearize it we square both sides.
        %
        \beq
            T^2 = \frac{4 \pi^2}{k} m
        \eeq

        Now the equation is linear between $T^2$ and $m$. A graph between these quantities will have a gradient/slope equal to $\displaystyle \frac{4 \pi^2}{k}$.

        Complete the table above by filling out the remaining columns. $T$ is the time period which is calculating by averaging the $T_{20}$ and dividing by 20 to get the time for a single oscillation.
        %
        \beq
            T = \frac{T_{20\_1} + T_{20\_2} + T_{20\_3}}{60}
        \eeq

        $\Delta T$ is the uncertainty associated with the time period $T$ and is calculated using the standard deviation of the three $T_{20}$.
        %
        \beq
            \Delta T = \frac{1}{20} \Delta T_{20}
        \eeq

        Finally the uncertainty $\Delta (T^2)$ in the squared quantity $T^2$ is calculated using the propagation of error equation for raising powers.
        %
        \beq
            \frac{\Delta (T^2)}{T^2} = 2 \frac{\Delta T}{T}
        \eeq

        Once the table is completed use it to plot a graph of $T^2$ vs $m$. Draw a best-fit line as well as steep and shallow lines.

        \begin{enumerate}[(i)]

            \item Calculate the value of the slope as well as its associated uncertainty.

            \item Use the value of the slope and its uncertainty to calculate the value of the spring constant and its associated uncertainty.

            \item Compare this value to the one calculated in section \ref{single_spring} by calculating the percentage difference (equation \eqref{percentage_difference}) between the values.

        \end{enumerate}
