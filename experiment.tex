\section*{Objective}

    Calculate the value of $g$ (acceleration due to gravity) using a simple pendulum.

\section*{Apparatus}

    \begin{itemize}

        \item String
        \item Mass (Bob)
        \item Stand
        \item Alligator Clip
        \item Stop-watch
        \item Meter-stick

    \end{itemize}

\section*{Procedure}

    A simple pendulum consists of a mass (known as bob) which is attached to a string whose other end is held fixed (the pivot). The bob is then free to swing to and fro about the pivot. The time-period of these oscillations is given by
    %
    \beq \label{main}
        T = 2 \pi \sqrt{\frac{l}{g}}
    \eeq
    %
    where
    %
    \begin{itemize}
        \item $T$ is the Time Period
        \item $l$ is the length of the pendulum (from the pivot to the center of the bob)
        \item $g$ is the acceleration due to gravity
    \end{itemize}

    Our aim is to vary the length $l$ of the pendulum, by changing the length of the string, and measuring the corresponding value of the Time Period $T$. These observations will be used to calculate the value of $g$.

\subsection*{Setup}

    Tie the bob to one end of the string. Make sure that the string is slightly longer than the height of the stand. Wrap the free end of the string around the top of the stand. Use the alligator clip to secure the top end of the string so that the length $l$ from the pivot to the center of the bob does not change as the pendulum oscillates.

\subsection*{Tabulation}

    Record your data in a table with the following format.

    \begin{center}
        \begin{tabular}{| c | c | c | c | c | c |}

            \hline
            $l$ (\si{\centi\metre}) & $t_{10_1}$ (\si{\second}) & $t_{10_2}$ (\si{\second}) & $t_{10_3}$ (\si{\second}) & $t_{10_4}$ (\si{\second}) & $t_{10_5}$ (\si{\second}) \\

            \hline
               & & & & & \\
            \hline

        \end{tabular}
    \end{center}

    \vspace{\baselineskip}
    For each value of $l$ we will measure the time for 10 oscillations, five times, giving us the five $t_{10_i}$. This table consists of the measurements we take in the experiment. From these measured values we calculate other derived quantities which will allow us to achieve our objective of calculating the value of $g$.

    \textit{It is recommended that you observe and right down all of the measurements first before you calculate the rest of the values. Using a pencil to write down the values will make it easy to fix inevitable mistakes.}
