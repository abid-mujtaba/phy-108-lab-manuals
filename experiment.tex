\section{Objective}

    Calculate the moment of inertia of a flywheel.

\section{Apparatus}

    \begin{itemize}

        \item Flywheel (mounted on a wall bracket)
        \item String
        \item Stopwatch
        \item Meter Rule
        \item Mass-Hanger set

    \end{itemize}

\section{Procedure}

    In this experiment we will be calculating the moment of inertia $I$ of a flywheel by using the Law of Conservation of Energy. We have shown that by equating the initial potential energy with the final Kinetic Energy and the Work Done against friction the moment of inertia can be calculated using the equation
    %
    \beq \label{I}
        I = \frac{N m}{N + n} \left( \frac{2 g h}{\omega^2} - r^2 \right)
    \eeq
    %
    where
    %
    \begin{itemize}

        \item $I$ is the moment of inertia
        \item $m$ is the total mass of the weight attached to the flywheel using the string
        \item $h$ is the initial height of the weight
        \item $r$ is the radius of the axle of the flywheel (about which the string is wound)
        \item $n$ is the number of turns of the flywheel needed to wind the string to raise the weight to its initial position
        \item $N$ is the number of turns the flywheel makes after the weight has detached and it is brought to rest by friction
        \item $\omega$ is the angular speed of the flywheel right after the weight detaches
        \item $t$ is the time it takes

    \end{itemize}

    \eline
    We will be calculating $n$ and $\omega$ using
    %
    \beq \label{n}
        n = \frac{h}{2 \pi r}
    \eeq
    %
    and
    %
    \beq \label{omega}
        \omega = \frac{4 \pi N}{t}
    \eeq

    Our aim is to vary the mass $m$ of the weight (mass-hanger) and measure the corresponding values of $N$ and $t$. These will be used to calculate the corresponding values of the moment of inertia $I$ of the flywheel.

    \subsection{Setup}

        You will be provided with a flywheel mounted on a wall-bracket. A peg will be attached to the axle of the flywheel which will allow you to hang a loop of string from it. Start by measuring the radius $r$ of the axle of the flywheel (using a vernier caliper).

        Attach a long piece of string to the mass-hanger (the mass-hanger plus any additional mass you add to it will comprise the weight that will drive the flywheel). Place the weight on the floor (below the flywheel) and measure out a length of string that reaches the axle of the flywheel. Create a loop in the top end of the string which is large enough to fit around the peg (without being too tight). Adjust the length of the string such that when the weight is on the floor the loop of string should be just falling of the peg.

        The idea is to place the loop on the peg and turn the flywheel such that the string winds around its axle causing the weight to rise off of the floor. Try to keep the string from overlapping on itself as you wind it on the axle. Keep winding until the weight is almost touching the axle. Use the meter-stick to measure he height $h$ of the weight above the floor.

        Now release the weight from rest. As the weight descends the attached string will unwind from the flywheel causing it to rotate. When the weight reaches the floor the string will run out and its top loop will fall of the peg causing the weight to detach from the flywheel.

        Since the flywheel will have by this time acquired an angular speed $\omega$ it will continue to rotate until it is brought to rest by friction in the bearings which support the axle.

    \subsection{Observations}

        First measure and note the values of $r$ and $h$. These will remain constant over the experiment so they only need to be measured once. The value of $n$ is calculated using equation \eqref{n}. We use the standard value for the acceleration due to gravity $g$.

        \def\cw{40pt}
        \begin{ctable}{| c | c || c | c |}
            \hline
            \tHW{\cw}{r}{\cm} & \tHW{\cw}{h}{\cm} & \makebox[\cw]{$n$} & \tH{g}{\cmpss}\\
            \hline
            &&& $980$\\
            \hline
        \end{ctable}

        For each value of $m$ (total mass of the weight/mass-hanger) our task is to measure the time $t$ and the number of turns $N$ that the flywheel makes from when the weight detaches (the string falls off) until the flywheel comes to rest. Repeat this measurement three times to get three values of $t_i$ and $N_i$. Record your results in a table with the following format

        \def\cw{25pt}
        \newcommand{\tW}[1]{\makebox[\cw]{#1}}
        \begin{ctable}{| c || c | c | c | c | c | c |}
            \hline
            \tHW{\cw}{m}{\si{\gram}} & \tW{$N_1$} & \tHW{\cw}{t_1}{\sec} & \tW{$N_2$} & \tHW{\cw}{t_2}{\sec} & \tW{$N_3$} & \tHW{\cw}{t_3}{\sec} \\
            \hline
            &&&&&&\\
            \hline
        \end{ctable}

        Perform these observations for at least four different values of $m$ and tabulate your results.
