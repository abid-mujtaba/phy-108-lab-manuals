\documentclass{article}

\usepackage{manual-style}
% We provide definition of commands that are used in all Lab Manuals

\newcommand{\manualtitle}[1]{\begin{center}{\Huge #1}\end{center}\vspace{2\baselineskip}}

\newcommand{\beq}{\begin{equation}}
\newcommand{\eeq}{\end{equation}}

\newcommand{\beqn}{\begin{equation*}}           % Equation environment without numbering
\newcommand{\eeqn}{\end{equation*}}

\newcommand{\beqsn}{\begin{equation*}\begin{aligned}}
\newcommand{\eeqsn}{\end{aligned}\end{equation*}}

\newcommand{\beqcn}{\begin{equation*}\begin{gathered}}
\newcommand{\eeqcn}{\end{gathered}\end{equation*}}

\renewcommand{\arraystretch}{1.5}       % Increase row height of all tables

\newcommand{\paren}[1]{\left( #1 \right)}       % An environment that places its contents inside a pair of parentheses

\newcommand{\imply}{\,\, \Rightarrow \,\,}



\renewcommand{\arraystretch}{1.5}


\begin{document}

   \manualtitle{Compound Pendulum}

   \section*{Objective}
   
      Calculate the value of $g$ (acceleration of gravity) and $L$ (the length of the compound pendulum).


   \section*{Apparatus}

      \begin{itemize}

         \item Slotted metal bar
         \item Suspension bracket
         \item Stop-watch
         \item Meter Rule
         \item Telescope

      \end{itemize}


   \section*{Procedure}

      A compound pendulum is a rigid body whose mass is not concentrated at one point and which is capable of oscillating about some fixed pivot (axis of rotation). In this experiment we will be studying the behavior of a uniform metallic bar acting as a compound pendulum. The time-period of the oscillations of a uniform bar is governed by the equation
      %
      \beq \label{eqn_T}
         T = 2 \pi \sqrt{\frac{L^2 + 12 \, l^2}{12 \, g l}}
      \eeq
      %
      where
      %
      \begin{itemize}
         \item $T$ is the time period
         \item $L$ is the total length of the bar
         \item $g$ is the acceleration of gravity
         \item $l$ is the distance from the center of mass of the bar to the pivot
      \end{itemize}

      Our aim is to vary $l$ by changing the location of the pivot, and for each value of $l$ observe the time period $T$. These observations will be used to calculate the acceleration of gravity and the total length of the bar.

      \subsection*{Setup}

         You will be provided a metallic bar with a number of holes/slots placed along its length. Its two ends will be labeled \textbf{A} and \textbf{B}. The center of mass of the bar will be indicated by a line drawn across its middle. The bar is to be suspended from the wall-mounted bracket using a set of pin and nuts.

         Choose the end of the bar labeled \textbf{A}. Pass the pin through the hole/slot closest to this end (furthest away from the center) and use the provided nuts to tighten it in place. Ensure that roughly the same amount of pin protrudes from both ends.

         Now suspend the bar from the wall-bracket using the pin. The pin will support the bar and will allow it to oscillate parallel to the wall in the vertical plain.

         Place the telescope on a stool and position it so that you can view the suspended bar through it. Adjust the eye-piece (by sliding it) to bring the bar in to focus. Rotate the telescope in place until the cross-hairs are diagonal (no longer aligned with the horizontal and vertical directions). We will use the telescope to count the oscillations of the bar.

      \subsection*{Tabulation}

         Record your data in a table with the following format.

         \begin{table}[h]
            \centering

            \begin{tabular}{| c | c | c | c || c || c | c |}
               %\centering

               \hline
               $l$ ($cm$) & $t_{10_1}$ ($s$) & $t_{10_2}$ ($s$) & $t_{10_3}$ ($s$) & $T$ ($s$) & $l^2$ (${cm}^2$) & $T^2 l$ (${cm}.s^2$) \\

               \hline
                  & & & & & & \\
               \hline

            \end{tabular}
         \end{table}
         %
         For each value of $l$ we will measure the time for 10 oscillations, three times giving us the three $t_{10_i}$. $T$ (the time for one oscillation) is calculated by averaging over the $t_{10_i}$ and dividing by 10, that is
         %
         \beq \label{calc_T}
            T = \frac{t_{10_1} + t_{10_2} + t_{10_3}}{30}
         \eeq

         The four columns on the left are the measurements we take in the experiment. The rest of the columns are calculated using these measured values. The two columns on the right ($l^2$ and $T^2 \, l$) are the values we will plot on a graph.

         \textit{It is recommended that you observe and right down all of the measurements (first four columns) first before you calculate the rest of the values. Using a pencil to write down the values will make it easy to fix inevitable mistakes.}

      \subsection*{Observation}

      Start with the pin placed in the top-most hole/slot (next to the end labeled \textbf{A}). Use the meter-stick to measure the distance from the Center of Mass of the bar to the center of the pin from which the bar is suspended. This is $l$. Note down this value in the table.

      Start the oscillations by pulling the bar a few degrees (less than 10) out of its stationary vertical position and letting go. Use the telescope to observe the bar swinging past its initial vertical position. This will allow you to count complete oscillations of the pendulum (an oscillation is completed every time the pendulum swings past the initial position moving in the same direction). Use the stop-watch to measure the time taken to complete 10 oscillations. This is $t_{10_1}$.

      Stop the pendulum and then start it swinging again. Take two more measurements of the time taken to complete 10 oscillations. These are $t_{10_2}$ and $t_{10_3}$. Note these values in the table.

      Move the pin to the next hole/slot, below the current one. Measure the new value of $l$ and repeat the above procedure to get the three values for the time taken by 10 oscillations.

      Keep moving the pin to the next hole/slot until you reach the center of the bar. You will now have your complete set of measurements.


   \section*{Calculations}

      Use equation (\ref{calc_T}) to calculate $T$ and complete the table.

      The time period of the oscillations of a rigid bar is given by equation (\ref{eqn_T}). This equation is not linear in the dependence of $T$ on $l$. We transform the equation to get a linear relationship. We start with the original equation.
%
      \beqn
         T = 2 \pi \sqrt{\frac{L^2 + 12 \, l^2}{12 \, g l}}
      \eeqn
      %
      We square both sides to remove the square-root on the RHS.
      %
      \beqsn
         T^2 = 4 \pi^2 \paren{ \frac{L^2 + 12 \, l^2}{12 \, g l} } \\[0.25\baselineskip]
         \imply T^2 = \frac{\pi^2}{3 \, g} \paren{ \frac{L^2 + 12 \, l^2}{12 \, g l} }
      \eeqsn
         



\end{document}
