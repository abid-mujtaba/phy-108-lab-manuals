\tikzsetnextfilename{freebody}      % We set the filename which will be used for the pdf created by the externalize library. The prefix is still applicable so build/freebody.pdf will be created
\begin{center}
   \begin{tikzpicture}

      %\helpgrid{-4}{-6}{3}{2}      % Draw a grid to help place objects. Comment out in final version.
      
      \def\b{-5.5}         % Variable containing the position of the bottom of the bar
      \def\hw{0.2cm}       % Half-width of the bar
      \def\r{0.05}         % Radius of the pin-hole circles in the bar
      \def\cy{-2.25}       % Vertical position of center of bar

      \coordinate (cm) at (-120:-\cy);       % Location of center of mass after rotation

      \draw [dashed] (0,1) -- (0, \b);       % Dashed vertical line representing the equilibrium position. Note use of the variable \b

      \draw [rotate around={-30:(0,0)}]      % Rotate the following constructed object (bar with holes) about the origin by -30 degrees

         (0,1) -- (0,\b)       % The thick line representing the bar

         [thick, fill]                     % These attributes will be applied to the circles that follow
         (0,0) circle [radius=\r]          % Draw a thick point to represent the pivot
         (0,\cy) circle [radius=\r]        % Draw a thick point to represent the Center of Mass
      ;

      % Show angle theta using an arc with an arrow
      \def\ar{0.75}         % Radius of arc used to define angle
      \draw [->,>=stealth] (0, -\ar) arc (-90:-120:\ar);           % Draw an arc with an arrow at the end for the angle
      \draw (-105:\ar+0.25) node {$\theta$};                       % Note the use of polar coordinates to place the variable theta

      \dimline[offset = -0.5cm, show extensions = false]{(cm)}{(0,0)}{$l$};

      % Draw weight vector
      \draw [->] (cm) -- ++ (0,-2);
      \draw (cm) ++ (0,-2.25) node {$m g$};

      % Draw arc and angle theta between weight vector and rod
      \draw [->,>=stealth] (cm) ++ (0,-\ar) arc (-90:-120:\ar);
      \draw (cm) ++ (-105:\ar+0.25) node {$\theta$};

      % Draw pivot forces
      \draw [->,>=stealth] (0,0) -- (1,0);
      \draw (1.3, 0) node {$F_{p x}$};

      \draw [->,>=stealth] (0,0) -- (0,1);
      \draw (0, 1.3) node [fill=white] {$F_{p y}$};
      

   \end{tikzpicture}
\end{center}
