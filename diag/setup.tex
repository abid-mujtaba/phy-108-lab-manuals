\tikzsetnextfilename{setup}

\newkeycommand{\cylinder}[scale=1,rotate=0]{
   
   \begin{scope}[scale=\commandkey{scale}, rotate=\commandkey{rotate}]

      \def\w{1}     % Width of Cylinder
      \def\h{5}     % Height of Cylinder
      \def\uw{0.2}      % Width of top-edge of cylinder

      \draw                         % Cylinder
         (0,0) -- (0,\h)
         (\w,0) -- (\w,\h)
         (0,0) -- (\w,0)

         (0,\h) -- (-\uw,\h)
         (\w,\h) -- (\w + \uw,\h)
      ;

   \end{scope}
}

\begin{center}
   \begin{tikzpicture}

      \cylinder

      %\helpgrid{-4}{-6}{3}{2}      % Draw a grid to help place objects. Comment out in final version.
      
      %\def\b{-5.5}         % Variable containing the position of the bottom of the bar
      %\def\hw{0.2cm}       % Half-width of the bar
      %\def\r{0.05}         % Radius of the pin-hole circles in the bar
      %\def\cy{-2.25}       % Vertical position of center of bar

      %\draw [dashed] (0,1) -- (0, \b);       % Dashed vertical line representing the equilibrium position. Note use of the variable \b

      %\draw [rotate around={-30:(0,0)}]      % Rotate the following constructed object (bar with holes) about the origin by -30 degrees

         %(-\hw, 1) rectangle (\hw, \b)       % The rectangle representing the outline of the order

         %\foreach \y in {0.75,0,...,\b}      % We use a foreach loop to draw the circles on the bar placing them 0.75 units apart with one at the pivot
            %{(0,\y) circle [radius=\r]}      % The foreach loop variable \y is used to position the center of the circles

         %(-\hw, \cy) -- (-\r, \cy)       % Horizontal marker for Center of Mass which consists of lines on either side of the pin-hole (circle)
         %(\r, \cy) -- (\hw, \cy)

         %(\hw+0.05, \cy-0.2) node[right] {CM}
      %;

      %\draw [thick] (0,0) circle [radius=\r];          % Draw a thicker circle to represent the pivot

      %% Show angle theta using an arc with an arrow
      %\def\ar{1}         % Radius of arc used to define angle
      %\draw [->,>=stealth] (0, -\ar) arc (-90:-120:\ar);           % Draw an arc with an arrow at the end for the angle
      %\draw (-100:\ar+0.25) node {$\theta$};                       % Note the use of polar coordinates to place the variable theta

      %\draw (\hw+0.1, 0) node[right] {pivot};          % Place label "pivot" next to the pivot

      %\dimline[offset = -0.5cm, delta=\hw]{(-120:-\cy)}{(0,0)}{$l$};
      %\dimline[offset = -1cm, delta=\hw]{(-120:-\b)}{(60:1)}{$L$};

   \end{tikzpicture}
\end{center}
