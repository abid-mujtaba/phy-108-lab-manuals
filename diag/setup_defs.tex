% The definitions used by the two setup diagrams

% Define the height of the cylinder. We define it outside because this is also used by the flask
\def\Ch{5}

% Define the cylinder as an embeddable drawing
\newdrawing{\cylinder}{

      \def\hw{0.5}      % Half width of cylinder
      \def\h{\Ch}         % Height of Cylinder
      \def\uw{0.2}      % Width of top-edge of cylinder

      \draw                         % Cylinder
         (-\hw,0) -- (-\hw,\h)
         (\hw,0) -- (\hw,\h)
         (-\hw,0) -- (\hw,0)

         (-\hw,\h) -- (-\hw-\uw,\h)
         (\hw,\h) -- (\hw + \uw,\h)
      ;
}

% Define the conical flask
\newdrawing{\flask}{

   \def\chw{0.1}       % Half-width of lower column
   \def\ch{0.75}           % Height of lower column

   \def\thw{0.8}        % Half-width of the top of the slanting bit
   \def\th{0.5}         % Height of the top of the slanting bit

   \coordinate (ol) at (-\chw, 0);
   \coordinate (or) at (\chw, 0);


   % We will place the origin horizontally center and vertically where the slanting part meets the lower column
   \draw
      (ol) -- (-\chw, -\ch)
      (or) -- (\chw, -\ch)

      (ol) -- (-\thw, \th)
      (or) -- (\thw, \th)
   ;
}
