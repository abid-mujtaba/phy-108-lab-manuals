Viscosity is a property of fluids (liquids and gases) which determines how much resistance is experienced by an object trying to move through the fluid. In this experiment we will use Stoke's Law and the concept of terminal velocity to determine the viscosity of glycerin.

\section*{Objective}

   Use Stoke's Law to derive an equation relating the viscosity $\eta$ of a fluid to the time $t$ it takes for a sphere to fall a distance $d$ through it.

\section*{Experimental Setup}

   The equipment consists of a long glass cylinder filled with glycerin. A graduated scale is attached to the cylinder which allows us to measure distances along the cylinder.
   
   Small metallic spheres are dropped from the top of the cylinder. The viscosity of the glycerin produces a drag force that will eventually result in terminal velocity being achieved by the falling sphere. The terminal velocity is calculated by measuring the time $t$ it takes the sphere to fall through a distance $d$ inside the fluid.

   \diag

\section*{Free-Body Diagram}

   A sphere falling through a fluid experiences three forces: its weight $W$, an upward buoyant force $F_B$ (Archimedes' Principle) and the drag force $F_D$ (Stoke's Law).

   \diag

\section*{Derivation}

   The three forces acting on the falling sphere are given by
   \beqc
      W = m g\\
      F_B = \rho \, V g\\
      F_D = 6 \pi \eta r v\\
   \eeqc

   The first two forces remain constant but the drag force increases in magnitude as the sphere speeds up since it is directly proportional to the velocity $v$. Initially, when the velocity of the sphere is low, the drag force is low as well and therefore the net downward force and acceleration are large. This causes the velocity to increase. As the velocity increases the drag force increases and in turn the net downward force and acceleration decrease. The velocity keeps increasing but at a slower rate. Eventually the drag force increases until it exactly balances the other two forces. The net downward force and acceleration become zero and the velocity becomes constant. This is known as the terminal velocity.

   Therefore, by definition, when the terminal velocity is achieved the net force on the sphere is zero, the three forces balance each other out. Looking at the free-body diagram this means
   \beq
      W = F_B + F_D
   \eeq
