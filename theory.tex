Hooke's Law is a physical principle that states that a spring stretched (extended) or compressed by some distance produces a restoring force which is directly proportional to said distance. Mathematically, if an extension $x$ is accompanied by a restoring force $F$ then they are related by the equation
%
\beq \label{hookes_law}
    F = k \, x
\eeq
%
where $k$ is the Spring Constant.

\section{Experimental Setup}

    \diag

    The weight of the mass attached to the bottom of the spring provides the force necessary to stretch it. The extension of the spring is measured using the pointer attached to the mass hanger. The position of the pointer is read off from the ruler placed behind the pointer.

\section{Balancing Forces}

    If a weight is attached to an initially unextended spring the force applied by the weight is unbalanced and causes the spring to begin stretching (extending). As the extension of the spring increases it produces a restoring force given by \eqref{hookes_law} whose purpose is to oppose the stretching and return the spring to its initial unstretched condition. When the spring has extended enough the restoring force exactly balances the weight of the mass and the system achieves equilibrium.

    This is demonstrated by a free-body diagram of the mass where the only two forces acting upon it are its weight and the restoring force exerted by the spring.

    \diag

    Therefore at equilibrium (when the mass-spring system is at rest) the two forces must be balanced and so
    \beqsn
        F - mg &= 0\\
        \imply kx - mg &= 0\\
        \imply kx &= mg\\
        \imply x &= \frac{g}{k} m\\
    \eeqsn

    The final equation gives the relation between the mass attached to the string and the extension it produces.

% \section{Single Spring}
%
%     An experiment investigating Hooke's Law usually starts with a vertically suspended spring with a mass-hanger attached. The weight of the hanger stretches the spring and the pointer attached to the hanger
