A flywheel is a mechanical device consisting of a massive object mounted on an axle such that it is free to rotate about an axis with minimal friction. It is used to store rotational kinetic energy which can be harnessed later when a quick or smooth source of energy is required.

The energy stored in a flywheel depends upon its angular speed and is moment of inertia as described by the equation
%
\beq
    K = \frac{1}{2} I \omega^2
\eeq
%

The moment of inertia, $I$, of a rotating object is the rotational analog of mass in linear motion. That is, the moment of inertia determines how easy or difficult it is to change the speed of a rotating object. This is evident from studying both the linear and rotational forms of Newton's Second Law of Motion. For linear motion the second law takes the form
%
\beq \label{F_ma}
    F = m a
\eeq

It is clear that for a constant force $F$ increasing the mass will decrease the acceleration, that is, the more massive an object the more difficult it is to change its velocity. For rotational motion the second law takes the form
%
\beq
    \tau = I \alpha
\eeq
%
where $\tau$ is the torque (turning effect of a force), $\displaystyle \alpha \equiv \frac{d \omega}{dt}$ is the angular acceleration and $I$ is the moment of inertia.

In complete analogy to equation \eqref{F_ma} it is clear that for constant torque $\tau$ increasing the moment of inertia $I$ will decrease the angular acceleration $\alpha$, that is, the more moment of inertia an object has the more difficult it is to change its angular velocity $\omega$.
