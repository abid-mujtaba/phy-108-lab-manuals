Hooke's Law is a physical principle that states that a spring stretched (extended) or compressed by some distance produces a restoring force which is directly proportional to said distance. Mathematically, if an extension $x$ is accompanied by a restoring force $F$ then they are related by the equation
%
\beq \label{hookes_law}
    F = k \, x
\eeq
%
where $k$ is the Spring Constant.

\section{Experimental Setup}

    \diag

    The weight of the mass attached to the bottom of the spring provides the force necessary to stretch it. The extension of the spring is measured using the pointer attached to the mass hanger. The position of the pointer is read off from the ruler placed behind the pointer.

\section{Balancing Forces}

    If a weight is attached to an initially unextended spring the force applied by the weight is unbalanced and causes the spring to begin stretching (extending). As the extension of the spring increases it produces a restoring force given by \eqref{hookes_law} whose purpose is to oppose the stretching and return the spring to its initial unstretched condition. When the spring has extended enough the restoring force exactly balances the weight of the mass and the system achieves equilibrium.

    This is demonstrated by a free-body diagram of the mass where the only two forces acting upon it are its weight and the restoring force exerted by the spring.

    \diag

    Therefore at equilibrium (when the mass-spring system is at rest) the two forces must be balanced and so
    \beqs \label{mgkx}
        F - mg &= 0\\
        \imply kx - mg &= 0\\
        \imply kx &= mg\\
        \imply x &= \frac{g}{k} m\\
    \eeqs

    The final equation gives the relation between the mass attached to the string and the extension it produces.

\section{Single Spring}

    Using equation \eqref{mgkx} one can calculate the spring constant $k$ of a spring my performing an experiment where one varies the attached mass $m$ and measures the corresponding extension $x$. The slope of the graph of $x$ vs. $m$ is then given by $g / k$.

\section{Springs in parallel}

    A mass can be suspended from two springs in parallel by connecting the bottoms of the two springs by a pole and suspending the mass from its center, as shown in the diagram below.

    \diag

    The weight of the mass causes both springs to be extended by the same amount $x$ since they are connected by means of the pole. The two springs are essentially responding to the weight of the mass by extending and applying a restoring force. Thus together they are behaving as a single spring would behave.

    The question is: What is the \textit{effective} spring constant of the two parallel springs? That is, for what value of the spring constant would a \textbf{single} spring have the same effect as the two springs in parallel.

    Let the two springs have spring constants $k_1$ and $k_2$. Let the effective spring constant (of the equivalent single spring) be $k$. If the application of mass $m$ to the system produces an extension $x$ we can use \eqref{mgkx} to relate $x$ to $m$ using the spring constants. For the single spring case the equation is straight-forward, it is simply \eqref{mgkx}.

    For the springs in parallel we draw a free-body diagram and equate the upward and downward forces.

    \diag

    Since the two springs in parallel have the same extension $x$ we have
    \beq \label{parallel}
        k_1 x + k_2 x = mg
    \eeq

    Since the parallel spring system and the effective single spring are by definition equivalent we can use \eqref{mgkx} to substitute for $mg$ in \eqref{parallel} giving us
    %
    \beqc
        k x = m g = k_1 x + k_2 x\\
        \imply k x = (k_1 + k_2) x\\
        \imply k = k_1 + k_2
    \eeqc

    Therefore the \textit{effective} spring constant of two springs in parallel is simply the sum of their individual spring constants. This basically indicates that it is harder to stretch two springs in parallel than it is to stretch either one by itself.
