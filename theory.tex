A flywheel is a mechanical device consisting of a massive object mounted on an axle such that it is free to rotate about an axis with minimal friction. It is used to store rotational kinetic energy which can be harnessed later when a quick or smooth source of energy is required.

The energy stored in a flywheel depends upon its angular speed and is moment of inertia as described by the equation
%
\beq
    K = \frac{1}{2} I \omega^2
\eeq
%

The moment of inertia, $I$, of a rotating object is the rotational analog of mass in linear motion. That is, the moment of inertia determines how easy or difficult it is to change the speed of a rotating object. This is evident from studying both the linear and rotational forms of Newton's Second Law of Motion. For linear motion the second law takes the form
%
\beq \label{F_ma}
    F = m a
\eeq

It is clear that for a constant force $F$ increasing the mass will decrease the acceleration, that is, the more massive an object the more difficult it is to change its velocity. For rotational motion the second law takes the form
%
\beq
    \tau = I \alpha
\eeq
%
where $\tau$ is the torque (turning effect of a force), $\displaystyle \alpha \equiv \frac{d \omega}{dt}$ is the angular acceleration and $I$ is the moment of inertia.

In complete analogy to equation \eqref{F_ma} it is clear that for constant torque $\tau$ increasing the moment of inertia $I$ will decrease the angular acceleration $\alpha$, that is, the more moment of inertia an object has the more difficult it is to change its angular velocity $\omega$.

\section{Objective}

    Use conservation of energy to calculate the moment of inertia of a flywheel.

\section{Experimental Setup}

    The flywheel consists of a metallic disc mounted on a cylindrical axle. The two ends of the axle are mounted on fixed supports (part of a wall bracket) using ball bearings so that the flywheel is free to rotate with minimal friction.

    On one side of the axle (halfway between the disc and the support) a small peg has been welded on. A looped end of a string is placed on this peg and then the string is wound on the axle such that unwinding the string causes the axle to rotate. A weight (mass hanger) is attached to the other end of the string.

    The length of the string is adjusted such that when all of the string is unwound the attached weight is just about to touch the floor. The procedure requires that initially all of the string is wound on the axle and the attached weight is almost touching the axle. The weight is then let go. It descends due to gravity, pulling the string down with it, which in turn causes the axle to rotate as the string on it unwinds.

    The string runs out just before the weight reaches the floor. Since the looped end of the string is simply placed on the peg and not tied tightly to it the loop simply falls of the peg when the weight reaches the floor. The flywheel is subsequently left free to continue rotating with the string now detached from it. This last requirement is important since it will allow us to calculate the Work Done due to friction in the ball bearings supporting the axle.

    The setup starts with the weight attached to the string and lying on the floor. The length of the string is adjusted so that the weight is on the floor, the string is taut (stretched), and the loop on the upper end of the string is barely on the peg.

    The flywheel is now rotated to wind the string on the axle (without overlapping) and thereby lift the weight up off of the ground. Keep winding the string until the top of the weight is almost touching the axle. Measure the height of the bottom of the weight above the ground. This is $h$ and is the amount by which the weight will descend when we let it go.

\section{Derivation}

    The measurement begins with us letting go of the weight from rest at a height $h$. The force of gravity will pull down on the weight making it descend. The string attached to the weight will cause the axle to rotate as it unwinds. Just before the weight hits the floor the string will run out, the loop will fall off of the peg, and the flywheel will continue to rotate freely after becoming detached from the weight.

    Let us relate the angular speed of the flywheel with the descent of the weight using the conservation of energy. Initially both the weight and the flywheel are stationary so the total energy comes only from the potential energy of the weight due to its height $h$.

    After the weight has descended all of this potential energy is converted in to kinetic energy in the weight and the flywheel along with a portion that is lost due to friction in the ball bearings (as the flywheel rotates).

    If we define $W_f$ as the work done by friction in the ball bearings during a \textbf{single} turn of the flywheel then conservation of energy gives us
    %
    \beq \label{energy_1}
        m g h = \frac{1}{2} m v^2 + \frac{1}{2} I \omega^2 + n W_f
    \eeq
    %
    where $m$ is the mass of the descending weight, $v$ is the final velocity of the weight (just before impact), $\omega$ is the final angular speed of the flywheel and $n$ is the number of turns the flywheel makes as the weight descends (and the string unwinds).

    Our purpose is to calculate the value of $I$ using this experiment. To that end we need to able to measure or calculate the value of every other variable in equation \eqref{energy_1}. We can measure $m$ and $h$ directly, and use the standard value of $g$, leaving us with $v$, $\omega$, $n$ and $W_f$ unknown.

    The first step is to reduce the number of unknown by relating $v$ (the velocity of the weight) to $\omega$ (the angular speed of the flywheel). Since the weight and the flywheel are connected by means of a taut string the distance through which the weight descends must be equal to the distance through which the rim of the axle rotates (that is if the weight descends by \quantity{1}{\cm} then the outer surface of the axle must necessarily rotate through \quantity{1}{\cm} to allow \quantity{1}{\cm} of string to unwind allowing the weight to descend).

    Mathematically this is stated as $y = s = r \theta$ where $y$ is the amount by which the weight descends which is equal to $s$ the amount through which the surface of the axle rotates which in turn is equal to $r \theta$ where $r$ is the radius of the flywheel, and $\theta$ is the corresponding angle through which the axle (and therefore the flywheel) rotates.

    Differentiating $y = r \theta$ gives us
    %
    \beq \label{v_rw}
        v = r \omega
    \eeq
    %
    that is the velocity of the falling weight is related to the angular speed of the flywheel by means of the radius of the axle. The number of turns $n$ that the flywheel makes as the weight descends can also be determined using the radius $r$. When the weight descends through $h$ the flywheel also turns through a distance equal to $h$ which corresponds to $n$ turns. Since each turn corresponds to a circumference we have
    %
    \beq \label{h_nr}
        h = n 2 \pi r \imply n = \frac{h}{2 \pi r}
    \eeq

    Substituting equations \eqref{v_rw} and \eqref{h_nr} in to equation \eqref{energy_1} will simplify it to
    %
    \beq \label{energy_2}
        m g h = \frac{1}{2} m r^2 \omega^2 + \frac{1}{2} I \omega^2 + \frac{h}{2 \pi r} W_f
    \eeq

    The next step is figuring out a means to measure or calculate the final angular speed $\omega$ of the flywheel. This can be related very elegantly with the problem of measuring $W_f$ (the work done per turn by friction). We assume that $W_f$ is constant, that is, it doesn't change with the angular speed of the flywheel. This assumption means we are taking the angular acceleration due to friction to be constant.

    After the weight has descended and the string has detached from the flywheel the only force acting on the flywheel is the friction in the bearings. This will cause the flywheel to slow down and eventually come to rest. We measure the number of turns $N$ and time $t$ it takes for the flywheel to come to rest after the weight has detached, that is the time and turns it takes to go from an angular speed of $\omega$ to zero.

    Since the flywheel is decelerated entirely by friction the rotational kinetic energy is converted to work done against friction (in $N$ turns) giving us
    %
    \beq \label{W_f}
        N W_f = \frac{1}{2} I \omega^2 \imply W_f = \frac{1}{2 N} I \omega^2
    \eeq
    %

    We can easily measure $N$ by counting the number of turns the flywheel makes as it decelerates (after the weight has detached). This now leaves us with measuring or calculating $\omega$.
