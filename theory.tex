% Section explaining the theory governing the experiment

A compound pendulum (also known as a physical pendulum) consists of a \textbf{rigid} body oscillating about a pivot. This experiment uses a uniform metallic bar with holes/slots cut down the middle at regular intervals. The bar can be hung from any one of these holes allowing us to change the location of the pivot.

\section{Objective}

Derive an equation for the time period $T$ of the oscillations of a uniform metallic bar suspended from a pivot passing through it.

\section{Experimental Setup}

   The experimental equipment consists of a thin uniform metallic bar with holes/slots placed through it at regular intervals. By allowing the bar to swing from different slots one can change the Moment of Inertia and consequently the Time Period of oscillations.

   \begin{center}
      \tikzsetnextfilename{setup}
\begin{center}
   \begin{tikzpicture}

       \def\r{5}       % Radius of the string. This definition is used in the code that we input from diag/setup-common below

       \input{diag/setup-common}

       \dimline[offset=-0.5cm, delta=0.5cm]{(bc)}{(0,0)}{$l$};

   \end{tikzpicture}
\end{center}
        % Include tex code for the diagram of the experimental setup
   \end{center}

   We define the total length of the bar as $L$ and the distance from the pivot to the center of mass (CM) of the bar to be $l$ as indicated in the diagram above. The position of the bar at any instant of time is given by the angle $\theta$. When allowed to swing the bar performs an approximation of simple harmonic motion, that is, the angle $\theta$ varies in a cyclic fashion with time period $T$.

\section{Free-Body Diagram}

   To calculate the time period $T$ one has to derive the equation of motion $\theta(t)$, namely how the angle $\theta$ varies as a function of time $t$. The first step, as always, is drawing the extended free body diagram of the system (extended because we are dealing with a rotational system and therefore the distance from the pivot is significant).

   \begin{center}
      \begin{tikzpicture}

   %\helpgrid{-4}{-6}{3}{2}      % Draw a grid to help place objects. Comment out in final version.
   
   \def\b{-5.5}         % Variable containing the position of the bottom of the bar
   \def\hw{0.2cm}       % Half-width of the bar
   \def\r{0.05}         % Radius of the pin-hole circles in the bar
   \def\cy{-2.25}       % Vertical position of center of bar

   \draw [dashed] (0,1) -- (0, \b);       % Dashed vertical line representing the equilibrium position. Note use of the variable \b

   \draw [rotate around={-30:(0,0)}]      % Rotate the following constructed object (bar with holes) about the origin by -30 degrees

      (0,1) -- (0,\b)       % The thick line representing the bar

      [thick, fill]                     % These attributes will be applied to the circles that follow
      (0,0) circle [radius=\r]          % Draw a thick point to represent the pivot
      (0,\cy) circle [radius=\r]        % Draw a thick point to represent the Center of Mass
   ;

   % Show angle theta using an arc with an arrow
   \def\ar{1}         % Radius of arc used to define angle
   \draw [->,>=stealth] (0, -\ar) arc (-90:-120:\ar);           % Draw an arc with an arrow at the end for the angle
   \draw (-100:\ar+0.25) node {$\theta$};                       % Note the use of polar coordinates to place the variable theta

   \dimline[offset = -0.5cm, show extensions = false]{(-120:-\cy)}{(0,0)}{$l$};

\end{tikzpicture}

   \end{center}


   Since we are only interested in the angle $\theta$ that represents the rotation of the bar about the pivot we calculate the torque about the pivot. This has the added advantage that it removes the unknown force $\vec{F}_p$ from consideration because the torque of a force that passes through the pivot is zero. So the only force that generates a torque about the pivot is the weight $m g$:
   \beq
      \tau = m g l \sin(\theta)
   \eeq

   The effect of this torque is to produce angular acceleration according the Newton's Second Law of Motion:
   \beq
      \tau = I \alpha
   \eeq
   where $I$ is the Moment of Inertia of the bar about the pivot and $\alpha = \frac{d^2 \theta}{dt^2}$ is its angular acceleration.

   \vspace{\baselineskip} 
   The next step is to calculate the Moment of Inertia $I$ of the bar about the pivot for a given value of $l$. For this we use the parallel axis theorem. If the mass of the bar is $m$ then $I$ is given by
   \beq
      I = I_{CM} + m \, l^2
   \eeq
   where $I_{CM}$ is the moment of inertia of the bar about its center of mass and $l$ is the distance from the pivot to the center of mass. In turn, $I_{CM}$ is calculated by considering the bar to have negligible width and uniform mass distribution.
   \beq
      I_{CM} = \frac{1}{12} \, m \, L^2
   \eeq
   where $L$ is the total length of the bar.
