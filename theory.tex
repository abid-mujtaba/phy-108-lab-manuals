A flywheel is a mechanical device consisting of a massive object mounted on an axle such that it is free to rotate about an axis with minimal friction. It is used to store rotational kinetic energy which can be harnessed later when a quick or smooth source of energy is required.

The energy stored in a flywheel depends upon its angular speed and is moment of inertia as described by the equation
%
\beq
    K = \frac{1}{2} I \omega^2
\eeq
%

The moment of inertia, $I$, of a rotating object is the rotational analog of mass in linear motion. That is, the moment of inertia determines how easy or difficult it is to change the speed of a rotating object. This is evident from studying both the linear and rotational forms of Newton's Second Law of Motion. For linear motion the second law takes the form
%
\beq \label{F_ma}
    F = m a
\eeq

It is clear that for a constant force $F$ increasing the mass will decrease the acceleration, that is, the more massive an object the more difficult it is to change its velocity. For rotational motion the second law takes the form
%
\beq
    \tau = I \alpha
\eeq
%
where $\tau$ is the torque (turning effect of a force), $\displaystyle \alpha \equiv \frac{d \omega}{dt}$ is the angular acceleration and $I$ is the moment of inertia.

In complete analogy to equation \eqref{F_ma} it is clear that for constant torque $\tau$ increasing the moment of inertia $I$ will decrease the angular acceleration $\alpha$, that is, the more moment of inertia an object has the more difficult it is to change its angular velocity $\omega$.

\section{Objective}

    Use conservation of energy to calculate the moment of inertia of a flywheel.

\section{Experimental Setup}

    The flywheel consists of a metallic disc mounted on a cylindrical axle. The two ends of the axle are mounted on fixed supports (part of a wall bracket) using ball bearings so that the flywheel is free to rotate with minimal friction.

    On one side of the axle (halfway between the disc and the support) a small peg has been welded on. A looped end of a string is placed on this peg and then the string is wound on the axle such that unwinding the string causes the axle to rotate. A weight (mass hanger) is attached to the other end of the string.

    The length of the string is adjusted such that when all of the string is unwound the attached weight is just about to touch the floor. The procedure requires that initially all of the string is wound on the axle and the attached weight is almost touching the axle. The weight is then let go. It descends due to gravity, pulling the string down with it, which in turn causes the axle to rotate as the string on it unwinds.

    The string runs out just before the weight reaches the floor. Since the looped end of the string is simply placed on the peg and not tied tightly to it the loop simply falls of the peg when the weight reaches the floor. The flywheel is subsequently left free to continue rotating with the string now detached from it. This last requirement is important since it will allow us to calculate the Work Done due to friction in the ball bearings supporting the axle.
