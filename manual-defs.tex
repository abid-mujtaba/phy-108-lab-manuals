% We provide definition of commands that are used in all Lab Manuals

% We customize the \maketitle command to make it generate the content that we want
\makeatletter       % We use this command because we want to refer to variables in our definition that contain the character @
\renewcommand{\maketitle}{
   \begin{center}
      {\Huge \@title}
   \end{center}
   \vspace{2\baselineskip}
}
\makeatother        % Stop treating @ as a normal character


\newcommand{\beq}{\begin{equation}}
\newcommand{\eeq}{\end{equation}}

\newcommand{\beqn}{\begin{equation*}}           % Equation environment without numbering
\newcommand{\eeqn}{\end{equation*}}

\newcommand{\beqsn}{\begin{equation*}\begin{aligned}}
\newcommand{\eeqsn}{\end{aligned}\end{equation*}}

\newcommand{\beqc}{\begin{equation}\begin{gathered}}
\newcommand{\eeqc}{\end{gathered}\end{equation}}

\newcommand{\beqcn}{\begin{equation*}\begin{gathered}}
\newcommand{\eeqcn}{\end{gathered}\end{equation*}}

\renewcommand{\arraystretch}{1.5}       % Increase row height of all tables

\newcommand{\paren}[1]{\left( #1 \right)}       % An environment that places its contents inside a pair of parentheses

\newcommand{\imply}{\,\, \Rightarrow \,\,}

