% We provide definition of commands that are used in all Lab Manuals

\makeatletter       % We use this command because we want to refer to variables in our definition that contain the character @

% We declare new variables which will be used in the title
\def\@course{}      % The variables with @ in them are global and are used inside the \maketitle macro
\def\@experimentnumber{}
\newcommand{\course}[1]{\def\@course{#1}}       % We declare the \course command to be a means of setting the value of the \@course variable so that \@course is hidden from the end-user
\newcommand{\experimentnumber}[1]{\def\@experimentnumber{#1}}

% We customize the \maketitle command to make it generate the content that we want
\renewcommand{\maketitle}{
   {\hfill {\@course} - Experiment - \@experimentnumber}\\       % The braces around \@course forces it to respect the space that follows it 
   \begin{center}
      {\Huge \@title}
   \end{center}
   \vspace{2\baselineskip}
}

\makeatother        % Stop treating @ as a normal character


\newcommand{\beq}{\begin{equation}}
\newcommand{\eeq}{\end{equation}}

\newcommand{\beqn}{\begin{equation*}}           % Equation environment without numbering
\newcommand{\eeqn}{\end{equation*}}

\newcommand{\beqsn}{\begin{equation*}\begin{aligned}}
\newcommand{\eeqsn}{\end{aligned}\end{equation*}}

\newcommand{\beqc}{\begin{equation}\begin{gathered}}
\newcommand{\eeqc}{\end{gathered}\end{equation}}

\newcommand{\beqcn}{\begin{equation*}\begin{gathered}}
\newcommand{\eeqcn}{\end{gathered}\end{equation*}}

\renewcommand{\arraystretch}{1.5}       % Increase row height of all tables

\newcommand{\paren}[1]{\left( #1 \right)}       % An environment that places its contents inside a pair of parentheses

\newcommand{\imply}{\,\, \Rightarrow \,\,}

